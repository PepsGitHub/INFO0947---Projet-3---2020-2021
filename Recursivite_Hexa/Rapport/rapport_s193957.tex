\documentclass[a4paper, 11pt, oneside]{article}

\usepackage[utf8]{inputenc}
\usepackage[T1]{fontenc}
\usepackage[french]{babel}
\usepackage{array}
\usepackage{shortvrb}
\usepackage{listings}
\usepackage[fleqn]{amsmath}
\usepackage{amsfonts}
\usepackage{fullpage}
\usepackage{enumerate}
\usepackage{graphicx}             % import, scale, and rotate graphics
\usepackage{subfigure}            % group figures
\usepackage{alltt}
\usepackage{url}
\usepackage{indentfirst}
\usepackage{eurosym}
\usepackage{listings}
\usepackage{color}
\usepackage[table,xcdraw,dvipsnames]{xcolor}

% Change le nom par défaut des listing
\renewcommand{\lstlistingname}{Extrait de Code}

% Change la police des titres pour convenir à votre seul lecteur
\usepackage{sectsty}
\allsectionsfont{\sffamily\mdseries\upshape}
% Idem pour la table des matière.
\usepackage[nottoc,notlof,notlot]{tocbibind}
\usepackage[titles,subfigure]{tocloft}
\renewcommand{\cftsecfont}{\rmfamily\mdseries\upshape}
\renewcommand{\cftsecpagefont}{\rmfamily\mdseries\upshape}

\definecolor{mygray}{rgb}{0.5,0.5,0.5}
\newcommand{\coms}[1]{\textcolor{MidnightBlue}{#1}}

\lstset{
    language=C, % Utilisation du langage C
    commentstyle={\color{MidnightBlue}}, % Couleur des commentaires
    frame=single, % Entoure le code d'un joli cadre
    rulecolor=\color{black}, % Couleur de la ligne qui forme le cadre
    stringstyle=\color{RawSienna}, % Couleur des chaines de caractères
    numbers=left, % Ajoute une numérotation des lignes à gauche
    numbersep=5pt, % Distance entre les numérots de lignes et le code
    numberstyle=\tiny\color{mygray}, % Couleur des numéros de lignes
    basicstyle=\tt\footnotesize,
    tabsize=3, % Largeur des tabulations par défaut
    keywordstyle=\tt\bf\footnotesize\color{Sepia}, % Style des mots-clés
    extendedchars=true,
    captionpos=b, % sets the caption-position to bottom
    texcl=true, % Commentaires sur une ligne interprétés en Latex
    showstringspaces=false, % Ne montre pas les espace dans les chaines de caractères
    escapeinside={(>}{<)}, % Permet de mettre du latex entre des <( et )>.
    inputencoding=utf8,
    literate=
  {á}{{\'a}}1 {é}{{\'e}}1 {í}{{\'i}}1 {ó}{{\'o}}1 {ú}{{\'u}}1
  {Á}{{\'A}}1 {É}{{\'E}}1 {Í}{{\'I}}1 {Ó}{{\'O}}1 {Ú}{{\'U}}1
  {à}{{\`a}}1 {è}{{\`e}}1 {ì}{{\`i}}1 {ò}{{\`o}}1 {ù}{{\`u}}1
  {À}{{\`A}}1 {È}{{\`E}}1 {Ì}{{\`I}}1 {Ò}{{\`O}}1 {Ù}{{\`U}}1
  {ä}{{\"a}}1 {ë}{{\"e}}1 {ï}{{\"i}}1 {ö}{{\"o}}1 {ü}{{\"u}}1
  {Ä}{{\"A}}1 {Ë}{{\"E}}1 {Ï}{{\"I}}1 {Ö}{{\"O}}1 {Ü}{{\"U}}1
  {â}{{\^a}}1 {ê}{{\^e}}1 {î}{{\^i}}1 {ô}{{\^o}}1 {û}{{\^u}}1
  {Â}{{\^A}}1 {Ê}{{\^E}}1 {Î}{{\^I}}1 {Ô}{{\^O}}1 {Û}{{\^U}}1
  {œ}{{\oe}}1 {Œ}{{\OE}}1 {æ}{{\ae}}1 {Æ}{{\AE}}1 {ß}{{\ss}}1
  {ű}{{\H{u}}}1 {Ű}{{\H{U}}}1 {ő}{{\H{o}}}1 {Ő}{{\H{O}}}1
  {ç}{{\c c}}1 {Ç}{{\c C}}1 {ø}{{\o}}1 {å}{{\r a}}1 {Å}{{\r A}}1
  {€}{{\euro}}1 {£}{{\pounds}}1 {«}{{\guillemotleft}}1
  {»}{{\guillemotright}}1 {ñ}{{\~n}}1 {Ñ}{{\~N}}1 {¿}{{?`}}1
}
\newcommand{\tablemat}{~}

%%%%%%%%%%%%%%%%% TITRE %%%%%%%%%%%%%%%%
% Complétez et décommentez les définitions de macros suivantes :
\newcommand{\intitule}{Récursivité et Elimination de la Récursivité}
\newcommand{\Prenom}{Peissone}
\newcommand{\Nom}{Dumoulin}
\newcommand{\matricule}{s193957}
% Décommentez ceci si vous voulez une table des matières :
% \renewcommand{\tablemat}{\tableofcontents}

%%%%%%%% ZONE PROTÉGÉE : MODIFIEZ UNE DES DIX PROCHAINES %%%%%%%%
%%%%%%%%            LIGNES POUR PERDRE 2 PTS.            %%%%%%%%
\title{INFO0947: \intitule}
\author{\textsc{\Prenom}~\textsc{\Nom}, \matricule}
\date{}

\begin{document}
\maketitle
\newpage
\tablemat
\newpage
%%%%%%%%%%%%%%%%%%%% FIN DE LA ZONE PROTÉGÉE %%%%%%%%%%%%%%%%%%%%

%%%%%%%%%%%%%%%% RAPPORT %%%%%%%%%%%%%%%
% Complétez les sections ci-dessous

\section{Notations}

Avant de commencer à parler de la formulation récursive, introduisons d'abord les notations utilisées par la suite:\\

\begin{itemize}
    \item $hexa$ $\equiv$ chaîne de charactères représentant un nombre hexadécimal
    \item $n$ $\equiv$ taille de la chaîne de charactères $hexa$
    \item $hexa\_dec\_rec$ $\equiv$ résultat de la conversion d'un nombre hexadécimal en sa traduction décimale.
\end{itemize}

\section{Formulation Récursive}\label{formulation}
%%%%%%%%%%%%%%%%%%%%%%%%%%%%%%%%
%
% Fournissez et discutez ici la formulation récursive du problème
%

\subsection{Cas de base}

Une chaîne de caractères vide ne pouvant pas  être convertie, le premier cas à considérer est le cas où l'on a un seul charactère à savoir $n = 1$. Afin d'obtenir le nombre décimal à partir de son homologue en hexa, il suffit juste d'utiliser la fonction convert(). Fonction permettant de convertir un nombre hexadécimal en un nombre décimal correspondant.\\

Si $n = 1$ :\\

$hexa\_dec\_rec(hexa, n) = convert(hexa[n - 1])$\\

\subsection{Cas récursif}

Pour procéder de manière récursive, on prend tous les cas possibles dans l'ordre croissant à partir du cas de base + 1. Autrement dit lorsque n est strictement supérieur à 1.\\

Si $n > 1$ :\\

$hexa\_dec\_rec(hexa, n) = convert(hexa[n - 1]) + 16 * hexa\_dec\_rec(hexa, n - 1)$\\

\subsection{Synthèse}
En faisant la synthèse du cas de base et du cas récursif, on obtient la formulation récursive de $hexa\_dec\_rec$:


$hexa\_dec\_rec(hexa, n)$ = $\left\{
    \begin{array}{ll}
        convert(hexa[n-1]) & \mbox{si n = 1}\\
        convert(hexa[n-1]) + 16 \times hexa\_dec\_rec(hexa, n-1) & \mbox{sinon}
    \end{array}
\right.$


\section{Spécification}\label{specification}
%%%%%%%%%%%%%%%%%%%%%%%%%
%
% Fournissez et discutez ici la spécification formelle de la fonction
% hexa_dec_rec()
%
L'énoncé nous dit que la fonction \texttt{hexa\_dec\_rec} est de type \texttt{unsigned int} et prend comme paramètres \texttt{hexa} et \texttt{n} avec \texttt{hexa} une chaîne de charactères (\texttt{char*}) et \texttt{n} sa taille (\texttt{int})

\subsection{PréCondition}

Comme vu précédemment dans le cas de base, $n$ ne peut être nul. Sachant que $n$ est un entier, il paraît naturel de dire que $n$ doit être strictement supérieur à 0.
De plus, on ne peut pas convertir une chaîne de charactères qui n'existe pas. De ce fait, on exprime $hexa$ $!=$ NULL. Ce qui nous donne la préCondition suivante:\\

PréCondition $\equiv$ $hexa$ $!=$ NULL $\wedge$ $n > 0$

\subsection{PostCondition}
En postCondition, nous voulons que $hexa$ et $n$ ne soient pas modifiés et que $hexa\_dec\_rec$ soit égal à la notation introduite précédemment à savoir $hexa\_dec\_rec(hexa, n)$
Celà se traduit par la postCondition suivante:\\

PostCondition $\equiv$ $hexa\_dec\_rec = hexa\_dec\_rec(hexa, n)$ $\wedge$ $hexa = hexa_0$ $\wedge$ $n = n_0$

\subsection{Résumé}

\begin{lstlisting}
//PréCondition : hexa != NULL, n > 0
//PostCondition : hexa\_dec\_rec = hexa\_dec\_rec(hexa, n) $\wedge$ hexa = hexa$_0$ $\wedge$ n = n$_0$
unsigned int hexa_dec_rec(char *hexa, int n);
\end{lstlisting}

\section{Construction Récursive}\label{recur}
%%%%%%%%%%%%%%%%%%%%%%%%%%%%%%%%%
%
% Fournissez et discutez ici la construction formelle (avec les assertions
% intermédiaires) de la fonction hexa_dec_rec()
%

\subsection{Programmation Défensive}
On vérifie que la précondition est respectée en interdisant à hexa d'être NULL et n ne peut être négatif

\begin{lstlisting}
unsigned int hexa_dec_rec(char *hexa, int n){
    assert(hexa != (void*)0 && n > 0);
    // \{PréCondition $\equiv$ $hexa \neq$ NULL $\wedge$ $n > 0$)\}
}
\end{lstlisting}

\newpage

\subsection{Cas de Base}
On gère le cas de base où $n = 1$ après s'être assuré que la préCondition est bien respectée.

\begin{lstlisting}
// \{PréCondition $\equiv$ $hexa \neq$ NULL $\wedge$  $n > 0$)\}
if(n == 1)
    // \{n = 1 $\wedge$ hexa = hexa$_0$ $\wedge$ n = n$_0$\}
    return convert(hexa[n - 1]);
    // \{hexa\_dec\_rec(hexa, n) = convert(hexa[n - 1]) $\wedge$ hexa = hexa$_0$ $\wedge$ n = n$_0$\}
    // \{$\implies$ PostCondition\}
}
\end{lstlisting}

\subsection{Cas Récursif}
Il y a un seul cas récursif, lorsque n est strictement supérieur à 1.
\{PréCondition$_{REC}$\} et \{PostCondition$_{REC}$\} sont respectivement la PréCondition et la PostCondition de l'appel récursif.

\begin{lstlisting}
else
    // \{hexa $\neq$ NULL $\implies$ hexa$_{REC}$ $\neq$ NULL $\wedge$ n > 1\}
    // \{$\implies$ PréCondition$_{REC}$\}
    return convert(hexa[n - 1]) + 16 * hexa_dec_rec(hexa, n - 1);
    // \{PostCondition$_{REC}$ $\equiv$ hexa\_dec\_rec = hexa\_dec\_rec(hexa, n) $\wedge$ n = n$_0$ $\wedge$ hexa = hexa$_0$\}
    // \{hexa\_dec\_rec = hexa\_dec\_rec(hexa, n) $\wedge$ n = n$_0$ $\wedge$ hexa = hexa$_0$\}
    // \{$\implies$ PostCondition\}
\end{lstlisting}

\subsection{Code complet}

\begin{lstlisting}
unsigned int hexa_dec_rec(char *hexa, int n){
  assert(hexa != (void*)0 && n > 0);//Précondition

  if(n == 1)
    return convert(hexa[n - 1]);//Cas de base
  else
    return convert(hexa[n - 1]) + 16 * hexa_dec_rec(hexa, n - 1);//Cas récursif
}
\end{lstlisting}
\newpage
\section{Traces d'Exécution}\label{traces}
%%%%%%%%%%%%%%%%%%%%%%%%%%%%%
%
% Fournissez et discutez ici les traces d'exécution de la fonction romain_rec()
% pour les exemples donnés dans l'énoncé
%

% L'exemple ci-dessous donne une trace d'exécution pour un exercice sur les
% Piles (cfr. GameCodes associés).  Inspirez-vous du code LaTeX pour produire
% vos traces d'exécution.

Voici les traces d'exécution concernant les 3 exemples du fichier main-hexadécimal.c

\subsection{hexa\_dec\_rec("27", 2)}

 \begin{tabular}{|c|}
 \\
 \\
 hexa\_dec\_rec("27", 2)\\
 \hline
 \end{tabular}~~
 \begin{tabular}{|c|}
 \\
 \\
 7 + 16 * 2 = 39\\
 \hline
 \end{tabular}~~
 
\subsection{hexa\_dec\_rec("A23", 3)}

 \begin{tabular}{|c|}
 \\
 \\
 hexa\_dec\_rec("A23", 3)\\
 \hline
 \end{tabular}~~
 \begin{tabular}{|c|}
 \\
 \\
 3 + 16 * (2 + 16 * 10) = 2595\\
 \hline
 \end{tabular}~~
 
\subsection{hexa\_dec\_rec("A78E", 4)}

 \begin{tabular}{|c|}
 \\
 \\
 hexa\_dec\_rec("A78E", 4)\\
 \hline
 \end{tabular}~~
 \begin{tabular}{|c|}
 \\
 \\
 14 + 16 * (8 + 16 * (7 + 16 * (10))) = 42894\\
 \hline
 \end{tabular}~~
 
\section{Complexité}\label{complexite}
%%%%%%%%%%%%%%%%%%%%%
%
% Fournissez et discutez ici la complexité théorique de la fonction
%  hexa_dec_rec()
%

On prendra la découpe suivante :
\begin{center}
  \includegraphics[width=15.9cm]{découpage_complexite.png}
\end{center}

\begin{itemize}
    \item Dans le cas où $n = 1$ :\\
    
    $T(n)$ est constant car $T(1) = a$\\
    \item Dans le cas où $n > 1$ :\\

    La fonction hexa\_dec\_rec(hexa, n) va s'appeler récursivement en décrémentant la valeur courante de n à chaque appel jusqu'à atteindre le cas de base.
    On a donc n - 1 appels récursifs.\\
    
    $T(n)$ est linéaire car $T(n) = T(n-1) * b$\\
\end{itemize}

En résumé, on a :

\begin{eqnarray*}
  T(n)=
  \begin{cases}
    a &\text{si n = 1} \\
    \text{T(n-1) * b} &\text{sinon}\\
  \end{cases}
\end{eqnarray*}

$T(n) = T(n-1) * b$ $\rightarrow$ $O(n)$\\

La complexité de la fonction hexa\_dec\_rec(hexa, n) est \textbf{linéaire}.

\section{Dérécursification}\label{derecur}
%%%%%%%%%%%%%%%%%%%%%%%%%%%%
%
% Fournissez et discutez ici la dérécursification de la fonction hexa_dec_rec()
% Attention, il n'est pas question ici de fournir un algorithme itératif mais
% bien d'éliminer la récursivité comme cela a été vu au cours.
% La solution doit être proposée en utilisant le pseudo-code vu au cours (et
% dans les GameCodes du Chapitre 9).
%
Pour procéder à la dérécursification, on va utiliser le pseudo langage qu'on a vu dans les gamecodes associés.

\subsection{Code récursif}

\lstset{escapeinside=//}
\begin{lstlisting}[language=Algol]
hexa_dec_rec(String hexa, int n):
  if(n=1)
    then 
      r /$\leftarrow$/ convert(hexa[n - 1]);
    else
      r /$\leftarrow$/ convert(hexa[n - 1]) + 16 * hexa_dec_rec(hexa, n - 1);
\end{lstlisting}

\subsection{Code dérécursivé}

\lstdefinelanguage{Custom}{%
language = Algol,
morekeywords = {until},
escapeinside = //,
}
\begin{lstlisting}[language=Custom]
hexa_dec_rec'(String hexa, int n):
    s /$\leftarrow$/ hexa;
    u /$\leftarrow$/ n;
    until u = 1 do
        s /$\leftarrow$/ s;
        u /$\leftarrow$/ u - 1;
    end
    r /$\leftarrow$/ convert'(hexa[u - 1]);
\end{lstlisting}

\end{document}
